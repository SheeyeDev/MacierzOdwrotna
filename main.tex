\documentclass[12pt]{article}
\usepackage{graphicx}
\usepackage{indentfirst}
\usepackage{polski}
\usepackage{amsmath}
\usepackage{amssymb}
\usepackage{dirtree}
\usepackage{listings}
\usepackage{hyperref}
\usepackage{minted}
\usepackage{xcolor}
\usepackage{subcaption}
\definecolor{LightGray}{gray}{0.9}
\usepackage{tocloft}
\usepackage{hyperref}
\hypersetup{
    colorlinks=true,
    linkcolor=blue
}
\newtheorem{example}{Przykład}

\title{Dokumentacja projektu zespołowego nr 3}
\author{Anna Ćwiklińska, Krystian Gronkowski, Ihor Malyi}
\date{Maj 2023}

\begin{document}
\lstset{basicstyle=\ttfamily, columns=fullflexible, upquote=true}
\renewcommand{\lstlistingname}{Listing}

\maketitle
\newpage
{\hypersetup{hidelinks}
\tableofcontents
}
\newpage
\section{Treść zadania}
Zaimplementować znajdowanie macierzy odwrotnej metodą Gaussa-Jordana (metoda powinna działać przynajmniej dla macierzy do stopnia 6).

\section{Teoretyczny opis metody}
Metoda Gaussa-Jordana jest jedną z technik rozwiązujących układy równań liniowych i znajdujących macierze odwrotne. Jej celem jest przekształcenie macierzy wejściowej do postaci macierzy jednostkowej poprzez elementarne operacje na wierszach. Główna idea metody polega na wykorzystaniu operacji elementarnych takich jak dodawanie wielokrotności jednego wiersza do innego w celu wyzerowania elementów pozostających pod i nad główną przekątną. Po wykonaniu tych operacji na macierzy wejściowej, otrzymuje się macierz jednostkową po lewej stronie, a macierz odwrotna znajduje się po prawej stronie.\\

Niech $A$ będzie macierzą kwadratową stopnia $n$.
\begin{enumerate}
    \item     Skonstruuj macierz rozszerzoną $[A|I]$, gdzie $I$ jest macierzą jednostkową stopnia $n$.
\item     Dla $k$ od $1$ do $n$ wykonuj następujące kroki:
\begin{enumerate}
    \item Jeśli element $a_{kk}$ jest równy zero, zamień wiersze $k$ i $j$ z poniższego kroku, gdzie $j$ jest najmniejszym indeksem większym lub równym $k$ taki, że $a_{jk} \neq 0$. Jeżeli taki indeks $j$ nie istnieje, macierz $A$ jest osobliwa, a metoda Gaussa-Jordana nie może być zastosowana.
\item Pomnóż $k$-ty wiersz przez $1/a_{kk}$, aby uzyskać jedynkę na przekątnej.
\item Dla każdego wiersza $i$ różnego od $k$, od $1$ do $n$, wykonaj operację elementarną odejmowania wielokrotności $a_{ik}$-krotności $k$-tego wiersza od $i$-tego wiersza, aby wyzerować element $a_{ik}$.
\end{enumerate}
    Otrzymasz macierz rozszerzoną w postaci $[I|B]$, gdzie $B$ jest macierzą odwrotną do macierzy $A$.
\end{enumerate}
Warto zauważyć, że jeśli macierz $A$ jest osobliwa, czyli nie ma odwrotności, to nie będzie możliwe wykonanie wszystkich kroków powyższej metody, ponieważ krok (2a) wymaga istnienia elementu niezerowego w każdym kroku.

\begin{example}
Szukamy macierzy odwrotnej do macierzy 3. stopnia:
$$
A = 
\left[
\begin{array}{rrr}
1 & 2 & 3 \\
1 & 1 & 0 \\
0 & 2 & 3 \\
\end{array}
\right]
$$

\begin{flalign*}
& \left[
\begin{array}{rrrrrrr}
1& 2 & 3 & \vrule & 1 & 0 & 0 \\
1 & 1 & 0 & \vrule & 0 & 1 & 0 \\
0 & 2 & 3 & \vrule & 0 & 0 & 1
\end{array} 
\right]
\begin{matrix}
    \sim\\
    w_2-w_1\\
\end{matrix}
\left[
\begin{array}{rrrrrrr}
1& 2 & 3 & \vrule & 1 & 0 & 0 \\
0 & -1 & -3 & \vrule & -1 & 1 & 0 \\
0 & 2 & 3 & \vrule & 0 & 0 & 1
\end{array} 
\right]
\\
\begin{matrix}
    \sim\\
    -1 w_2
\end{matrix}
& \left[
\begin{array}{rrrrrrr}
1& 2 & 3 & \vrule & 1 & 0 & 0 \\
0 & 1 & 3 & \vrule & 1 & -1 & 0 \\
0 & 2 & 3 & \vrule & 0 & 0 & 1
\end{array} 
\right]
\begin{matrix}
\\
    \sim\\
    w_1 - 2 w_2\\
    w_3 - 2 w_2 
\end{matrix}
\left[
\begin{array}{rrrrrrr}
1& 0 & -3 & \vrule & -1 & 2 & 0 \\
0 & 1 & 3 & \vrule & 1 & -1 & 0 \\
0 & 0 & -3 & \vrule & -2 & 2 & 1
\end{array} 
\right]
\\
\begin{matrix}
    \sim\\
    -\frac{1}{3}w_3\\
\end{matrix}
& \left[
\begin{array}{rrrrrrr}
1& 0 & -3 & \vrule & -1 & 2 & 0 \\
0 & 1 & 3 & \vrule & 1 & -1 & 0 \\
0 & 0 & 1 & \vrule & \frac{2}{3} & -\frac{2}{3} & -\frac{1}{3}
\end{array} 
\right]
\begin{matrix}
\\
    \sim\\
    w_1 + 3 w_3\\
    w_2 - 3 w_3 
\end{matrix} 
\left[
\begin{array}{rrrrrrr}
1& 0 & 0 & \vrule & 1 & 0 & -1 \\
0 & 1 & 0 & \vrule & -1 & 1 & 1 \\
0 & 0 & 1 & \vrule & \frac{2}{3} & -\frac{2}{3} & -\frac{1}{3}
\end{array} 
\right]
\end{flalign*}

Zatem rozwiązaniem jest macierz:
$$
A^{-1} = 
\left[
\begin{array}{rrr}
1 & 0 & -1 \\
-1 & 1 & 1 \\
\frac{2}{3} & -\frac{2}{3} & -\frac{1}{3} \\
\end{array}
\right]
$$
\end{example}
\section{Opis programu}
Celem programu jest wyznaczenie macierzy odwrotnej macierzy podanej przez użytkownika za pomocą metody Gaussa-Jordana.Program jest napisany w języku Python3. Nie korzysta z zewnętrznych bibliotek, importuje tylko bibliotekę standardową Pythona \verb|math|, która jest używana w pliku \verb|get_data| do wykonywania operacji matematycznych.

\section{Opis algorytmu}
\begin{enumerate}
\item Początek programu znajduje się w funkcji \verb|main()|. W pierwszej kolejności użytkownik informowany jest o dokładności obliczeń.
\item Program importuje dane do macierzy przy użyciu funkcji \verb|import_data| z pliku \verb|getData.py|. Zaimportowane dane są przekazywane do konstruktora obiektu klasy \verb|Matrix|.
\item Konstruktor klasy \verb|Matrix| inicjalizuje macierz wejściową i macierz odwrotną. Macierz odwrotna jest inicjalizowana jako macierz jednostkowa.
\item Metoda \verb|findInverseMatrix()| znajduje macierz odwrotną. 
\begin{enumerate}
    \item Najpierw wywoływana jest metoda \verb|checkSwaps()|, która sprawdza, czy występują zera na diagonalnej macierzy wejściowej i w~razie potrzeby zamienia wiersze za pomocą metody \verb|swapRows()|. Jeśli nie jest możliwe ustalenie macierzy odwrotnej, program wypisuje komunikat i kończy działanie.
    \item Następnie, przy użyciu pomocniczej metody \verb|divideRow()|, dla każdego wiersza macierzy wejściowej dokonywany jest podział wiersza przez element diagonalny, aby przekształcić go w jedynkę.
    \item Dla każdego wiersza różnego od aktualnego, za pomocą metody \verb|substractRow()| aktualny wiersz jest odejmowany od pozostałych wierszy odpowiednią ilość razy, aby przekształcić ich elementy  w~zera.
\end{enumerate}
\item W funkcji \verb|main()| wywoływana jest metoda \verb|printInverseMatrix()|, która wypisuje macierz odwrotną w sformatowanej formie.
\end{enumerate}

\section{Instrukcja użytkowania}
\subsection{Dane wejściowe}
\subsection{Dane wyjściowe}
\section{Struktura plików}
\dirtree{%
    .1 MacierzOdwrotna.
    .2 data.txt.
    .2 getData.py.
    .2 main.py.
    .2 matrix.py.
}
\subsection{Opis struktury plików}
\begin{itemize}
    \item Folder \verb|MacierzOdwrotna| zawiera wszystkie pliki niezbędne do funkcjonowania programu.
    \item Plik \verb|data.txt| jest plikiem tekstowym, który przeznaczony jest do przekazania danych wejściowych do programu. Plik ten zawiera macierz kwadratową, której odwrotność ma za zadanie wyznaczyć program.
    \item Plik \verb|getData.py| odpowiada za pobranie danych z pliku \verb|data.txt|. Zawiera funkcję \verb|import_data()|, która otwiera plik z danymi i pobiera z niego macierz wejściową, sprawdzając przy tym poprawność wartości.
    \item Plik \verb|main.py|, który odpowiada za główne wykonanie programu.
    \item \verb|matrix.py| zawiera klasę \verb|Matrix|, w której zdefiniowane są funkcje: \begin{itemize}
        \item \verb|findInverseMatrix()|
        \item \verb|checkSwaps()|
        \item \verb|swapRows()| \verb|swap1| \verb|swap2|
        \item \verb|printInverseMatrix()|
        \item \verb|substractRow()|, \verb|subFrom|, \verb|tosub|, \verb|howManyTimes|
        \item \verb|divideRow()|, która przyjmuje argumenty \verb|divideFrom| oraz \verb|divider|. Funkcja ta odpowiada za dzielenie danego wiersza macierzy przez stałą
    \end{itemize}
\end{itemize}

\section{Raport z demonstracji}
% \label{sec:raport}
% Po uruchomieniu programu w terminalu wyświetla się menu, jak pokazano na Rys. \ref{fig:menu}.

% \begin{figure}[H]
% \centering
% \includegraphics[width=0.5\textwidth]{menu.png}
% \caption{Menu programu}
% \label{fig:menu}
% \end{figure}
\section{Wnioski}
\section{Kod programu}
\subsection{get\_data.py}
\begin{minted}[breaklines]{python}
def import_data():
    data = []
    try:
        with open("./data.txt") as f:
            for line in f:
                line = line.strip()
                if not line:
                    continue
                values = line.split()
                row = []
                for x in range(len(values)):
                    try:
                        if abs(float(values[x])) > math.pow(10, 20):
                            print(
                                f"Liczba musi byc > {math.pow(10, 20)}. Błąd przy odczytie liczby {values[x]}"
                            )
                            exit()
                        row.append(float(values[x]))
                    except ValueError:
                        print(
                            f"Błąd: nie można przekonwertować wartości {values[x]} na liczby całkowite!"
                        )
                        exit()
                data.append(row)
    except FileNotFoundError:
        print("Błąd: plik nie istnieje!")
        exit()
    rows_count = len(data)
    for row in data:
        if len(row) != rows_count:
            print(
                f"Macierz nie jest kwadratowa. Liczba wierszy {rows_count}, liczba kolumn {len(row)}."
            )
            exit()

    return data

\end{minted}
\subsection{main.py}
\begin{minted}[breaklines]{python}
    def main():
        print("Dokładność do 7 liczby po przecinku.")
        matrix = Matrix(import_data())
        matrix.findInverseMatrix()
        matrix.printInverseMatrix()


if __name__ == "__main__":
    main()
\end{minted}
\subsection{matrix.py}
\begin{minted}[breaklines]{python}
class Matrix:
    def __init__(self, matrix) -> None:
        self.matrix = matrix
        self.inverse = [[0 for x in range(len(matrix[0]))] for x in range(len(matrix))]
        for x in range(len(self.inverse)):
            for y in range(len(self.inverse[x])):
                if x != y:
                    self.inverse[x][y] = 0
                else:
                    self.inverse[x][y] = 1

    def findInverseMatrix(self):
        self.checkSwaps()
        for x in range(len(self.matrix)):
            self.divideRow(x, 1 / self.matrix[x][x])
            for y in range(len(self.matrix)):
                if y != x:
                    self.substractRow(y, x, self.matrix[y][x] / self.matrix[x][x])

    def checkSwaps(self):
        for x in range(len(self.matrix)):
            if self.matrix[x][x] == 0:
                for y in range(len(self.matrix)):
                    if self.matrix[y][x] != 0 and self.matrix[x][y] != 0:
                        self.swapRows(x, y)
                        break
                    elif y == len(self.matrix) - 1:
                        print(
                            "Dla podanej macierzy nie da się ustalić macierzy odwrotnej."
                        )
                        exit()

    def swapRows(self, swap1, swap2):
        for x in range(len(self.matrix[swap1])):
            tmp = self.matrix[swap1][x]
            self.matrix[swap1][x] = self.matrix[swap2][x]
            self.matrix[swap2][x] = tmp

    def printInverseMatrix(self):
        pattern = "{:>25.7g} " * len(self.inverse[0])
        for row in self.inverse:
            print(pattern.format(*row))

    def substractRow(self, subFrom, tosub, howManyTimes):
        for x in range(len(self.matrix[subFrom])):
            self.matrix[subFrom][x] -= self.matrix[tosub][x] * howManyTimes
            self.inverse[subFrom][x] -= self.inverse[tosub][x] * howManyTimes

    def divideRow(self, divideFrom, divider):
        for x in range(len(self.matrix[divideFrom])):
            self.matrix[divideFrom][x] *= divider
            self.inverse[divideFrom][x] *= divider
\end{minted}
\section*{Bibliografia}
\begin{thebibliography}{}
	
\bibitem{Jaruszewska-Walczak}
Jaruszewska-Walczak, D. Wykład z Algorytmów numerycznych.

\bibitem{Kincaid}
Kincaid, D., \& Cheney, W. \emph{Analiza numeryczna}.

\bibitem{Tatjewski}
Tatjewski, P. \emph{Metody numeryczne}
	
\end{thebibliography}
\end{document}
